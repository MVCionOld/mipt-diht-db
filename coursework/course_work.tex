\documentclass{article}
\usepackage[utf8]{inputenc}
\usepackage[12pt]{extsizes} 
\usepackage[T2A]{fontenc}
\usepackage[russian]{babel}
\usepackage[left=20mm, top=15mm, right=15mm, bottom=15mm, nohead, footskip=10mm]{geometry}
\usepackage{setspace,amsmath}
\usepackage{enumerate}
\usepackage{graphicx}
\usepackage{booktabs}

\graphicspath{}
\DeclareGraphicsExtensions{.pdf,.png,.jpg}


\author{Михаил Цион}
\date{Апрель-май 2018}
\title{Курсач БД}

\begin{document}


\begin{center}
    \hfill \break
    \large{МИНОБРНАУКИ РОССИИ}\\
    \small{\textbf{МОСКОВСКИЙ ФИЗИКО-ТЕХНИЧЕСКИЙ ИНСТИТУТ \\(ГОСУДАРСТВЕННЫЙ УНИВЕРСИТЕТ)}}\\
    \hfill \break
    \normalsize{Физтех-школа прикладной математики и информатики (ФПМИ)}\\
    \normalsize{Факультет инноваций и высоких технологий (ФИВТ)}\\
    \hfill \break
    \hfill \break
    \hfill \break
    \hfill \break
    \large{ПРОЕКТИРОВАНИЕ БАЗЫ ДАННЫХ КАРШЕРИНГОВОЙ КОМПАНИИ}\\
    \hfill \break
    \hfill \break
    \hfill \break
    \hfill \break
    \hfill \break
    \normalsize{Курсовой проект}\\
    \hfill \break
    \hfill \break
    \normalsize{Выполнил:}\\
    \normalsize{Цион Михаил Владиславович}\\
    \normalsize{Группа 797}\\
    \hfill \break
    \hfill \break
\end{center}
\vfill
\begin{center} Москва 2018 \end{center}


\thispagestyle{empty}
\newpage

\section{Концептуальная модель}

Основные сущности:

\begin{itemize}
    \item
        Vehicle
            
    \item
        Location
            
    \item
        Employee 
            
    \item
        Client
        
    \item
        Rental
        
\end{itemize}

Отношения между сущностями приведены в схеме ниже:

\begin{figure}[h]
    \center{\includegraphics[scale=0.42]{scheme.jpg}}
    \caption{Концептуальная модель}
    \label{fig:image}
\end{figure}

\newpage

\section{Логическая модель}

Проектируемая база находится в 3НФ. 

\begin{figure}[h]
    \center{\includegraphics[scale=0.5]{logical.jpg}}
    \caption{Логическая модель (ER-диаграмма)}
    \label{fig:image}
\end{figure}

\newpage

\section{Физическая модель}


\begin{table}[h]
\centering
\resizebox{\textwidth}{!}{%
\begin{tabular}{|l|l|c|c|c|c|}
\hline
\multicolumn{6}{|c|}{\textbf{PARTICIPANT}} \\ \hline
\multicolumn{1}{|c|}{\textbf{Название}} & \multicolumn{1}{c|}{\textbf{Описание}} & \textbf{Тип данных} & \textbf{Ограничение} & \textbf{PK} & \textbf{FK} \\ \hline
participant\_id & Уникальный ключ сущности участник & INT &  & + &  \\ \hline
participant\_type\_code & Клиент или работник & INT & NOT NULL &  &  \\ \hline
first\_nm & Имя & VARCHAR(30) & NOT NULL &  &  \\ \hline
middle\_nm & Второе имя/отчество & VARCHAR(30) &  &  &  \\ \hline
last\_nm & Фамилия & VARCHAR(30) & NOT NULL &  &  \\ \hline
gender\_code & Пол & INT &  &  &  \\ \hline
email\_txt & Адрес электронной почты & VARCHAR(50) & * &  &  \\ \hline
phone\_no & Номер телефона & VARCHAR(50) & ** &  &  \\ \hline
\end{tabular}%
}
\caption{Сущность Participant/Участник}
\label{my-label}
\end{table}


\begin{table}[h]
\centering
\resizebox{\textwidth}{!}{%
\begin{tabular}{|l|l|c|c|c|c|}
\hline
\multicolumn{6}{|c|}{\textbf{CLIENT\_PARTICIPANT}} \\ \hline
\multicolumn{1}{|c|}{\textbf{Название}} & \multicolumn{1}{c|}{\textbf{Описание}} & \textbf{Тип данных} & \textbf{Ограничение} & \textbf{PK} & \textbf{FK} \\ \hline
participant\_id & Уникальный ключ сущности клиент & INT &  & + & + \\ \hline
penalties\_cnt & Штрафные баллы & INT & NOT NULL, DEFAULT 0, \textgreater{}= 0 &  &  \\ \hline
departures\_cnt & Количество выполненных выездов & INT & NOT NULL, DEFAULT 0, \textgreater{}= 0 &  &  \\ \hline
\end{tabular}%
}
\caption{Сущность Client Participant/Клиент}
\label{my-label}
\end{table}


\begin{table}[h]
\centering
\resizebox{\textwidth}{!}{%
\begin{tabular}{|l|l|c|c|c|c|}
\hline
\multicolumn{6}{|c|}{\textbf{EMPLOYEE\_PARTICIPANT}} \\ \hline
\multicolumn{1}{|c|}{\textbf{Название}} & \multicolumn{1}{c|}{\textbf{Описание}} & \textbf{Тип данных} & \textbf{Ограничение} & \textbf{PK} & \textbf{FK} \\ \hline
participant\_id & Уникальный ключ сущности работник & INT &  & + & + \\ \hline
contract\_no & Номер контракта & VARCHAR(20) & NOT NULL, UNIQUE &  &  \\ \hline
insurance\_no & Номер страховки & VARCHAR(20) & NOT NULL, UNIQUE &  &  \\ \hline
salary\_amt & Зарплата & REAL & NOT NULL, DEFAULT 0.0, \textgreater{}= 0.0 &  &  \\ \hline
position\_code & Должность сотрудника & INT & NOT NULL &  &  \\ \hline
\end{tabular}%
}
\caption{Сущность Employee Participant/Работник}
\label{my-label}
\end{table}


\begin{table}[h!]
\centering
\resizebox{\textwidth}{!}{%
\begin{tabular}{|l|l|c|c|c|c|}
\hline
\multicolumn{6}{|c|}{\textbf{VEHICLE}} \\ \hline
\multicolumn{1}{|c|}{\textbf{Название}} & \multicolumn{1}{c|}{\textbf{Описание}} & \textbf{Тип данных} & \textbf{Ограничение} & \textbf{PK} & \textbf{FK} \\ \hline
vehicle\_id & Уникальный ключ сущности тр. средства & INT &  & + &  \\ \hline
registration\_plate\_no & Регистрационные знаки тр. средства & VARCHAR(20) & NOT NULL, UNIQUE &  &  \\ \hline
make\_nm & Марка & VARCHAR(30) &  &  &  \\ \hline
model\_nm & Модель & VARCHAR(30) &  &  &  \\ \hline
vin & Идентификационный номер тр. средства & VARCHAR(17) & NOT NULL, UNIQUE &  &  \\ \hline
electric\_flg & Является ли электромобилем & BOOLEAN & NOT NULL, DEFAULT FALSE &  &  \\ \hline
transmission\_code & Тип коробки передачи & INT & NOT NULL &  &  \\ \hline
departures\_cnt & Количество выполненых выездов & INT & NOT NULL, DEFAULT 0, \textgreater{}= 0 &  &  \\ \hline
\end{tabular}%
}
\caption{Сущность Vehicle/Транспортное средство}
\label{my-label}
\end{table}


\begin{itemize}
    \item
        * - regexp\_match('\textasciicircum{}({[}\textbackslash{}w\textbackslash{}.\textbackslash{}-{]}+)@({[}\textbackslash{}w\textbackslash{}-{]}+)((\textbackslash{}.(\textbackslash{}w)\{2,3\})+)\$')\newline
    \item
        ** - regexp\_match('\textasciicircum{}((8|\textbackslash{}+7){[}\textbackslash{}- {]}?)?(\(?\d{3}\)?{[}\textbackslash{}- {]}?)?{[}\textbackslash{}d\textbackslash{}- {]}\{7,10\}\$')
\end{itemize}


\begin{table}[h!]
\centering
\resizebox{\textwidth}{!}{%
\begin{tabular}{|l|l|c|c|c|c|}
\hline
\multicolumn{6}{|c|}{\textbf{PARTICIPANT\_X\_VEHICLE}} \\ \hline
\multicolumn{1}{|c|}{\textbf{Название}} & \multicolumn{1}{c|}{\textbf{Описание}} & \textbf{Тип данных} & \textbf{Ограничение} & \textbf{PK} & \textbf{FK} \\ \hline
participant\_id & Внешний ключ сущности участник & INT &  & + & + \\ \hline
vehicle\_id & Внешний ключ сущности тр. средства & INT &  & + & + \\ \hline
\end{tabular}%
}
\caption{Сущность Participant-Vehicle}
\label{my-label}
\end{table}


\begin{table}[h!]
\centering
\resizebox{\textwidth}{!}{%
\begin{tabular}{|l|l|c|c|c|c|}
\hline
\multicolumn{6}{|c|}{\textbf{LOCATION}} \\ \hline
\multicolumn{1}{|c|}{\textbf{Название}} & \multicolumn{1}{c|}{\textbf{Описание}} & \textbf{Тип данных} & \textbf{Ограничение} & \textbf{PK} & \textbf{FK} \\ \hline
location\_id & Уникальный ключ сущности местность & INT &  & + &  \\ \hline
partner\_id & Внешний ключ сущности партнёра & INT &  &  & + \\ \hline
latitude & Широта & REAL & NOT NULL &  &  \\ \hline
longitude & Долгота & REAL & NOT NULL &  &  \\ \hline
address\_txt & Адрес & VARCHAR(20) &  &  &  \\ \hline
city\_nm & Город & \multicolumn{1}{l|}{VARCHAR(20)} & \multicolumn{1}{l|}{} & \multicolumn{1}{l|}{} & \multicolumn{1}{l|}{} \\ \hline
state\_nm & Область/штат & \multicolumn{1}{l|}{VARCHAR(20)} & \multicolumn{1}{l|}{} & \multicolumn{1}{l|}{} & \multicolumn{1}{l|}{} \\ \hline
type\_code & Тип объекта на местности: стоянка/СТО/гараж/... & INT &  &  &  \\ \hline
\end{tabular}%
}
\caption{Сущность Location/Местность}
\label{my-label}
\end{table}


\begin{table}[h!]
\centering
\resizebox{\textwidth}{!}{%
\begin{tabular}{|l|l|c|c|c|c|}
\hline
\multicolumn{6}{|c|}{\textbf{PARTNER}} \\ \hline
\multicolumn{1}{|c|}{\textbf{Название}} & \multicolumn{1}{c|}{\textbf{Описание}} & \textbf{Тип данных} & \textbf{Ограничение} & \textbf{PK} & \textbf{FK} \\ \hline
partner\_id & Уникальный ключ сущности партнёра & INT &  & + &  \\ \hline
partner\_nm & Название партнерской компании & VARCHAR(50) & NOT NULL &  &  \\ \hline
specialization\_code & Специализация компании & INT &  &  &  \\ \hline
\end{tabular}%
}
\caption{Сущность Partner/Партнер}
\label{my-label}
\end{table}


\begin{table}[h!]
\centering
\resizebox{\textwidth}{!}{%
\begin{tabular}{|l|l|c|c|c|c|}
\hline
\multicolumn{6}{|c|}{\textbf{VEHICLE\_X\_LOCATION}} \\ \hline
\multicolumn{1}{|c|}{\textbf{Название}} & \multicolumn{1}{c|}{\textbf{Описание}} & \textbf{Тип данных} & \textbf{Ограничение} & \textbf{PK} & \textbf{FK} \\ \hline
vehicle\_id & Внешний ключ сущности тр. средства & INT &  & + & + \\ \hline
location\_id & Внешний ключ сущности местность & INT &  & + & + \\ \hline
\end{tabular}%
}
\caption{Сущность Vehicle-Location}
\label{my-label}
\end{table}


\begin{table}[h!]
\centering
\resizebox{\textwidth}{!}{%
\begin{tabular}{|l|l|c|c|c|c|}
\hline
\multicolumn{6}{|c|}{\textbf{RENTAL}} \\ \hline
\multicolumn{1}{|c|}{\textbf{Название}} & \multicolumn{1}{c|}{\textbf{Описание}} & \textbf{Тип данных} & \textbf{Ограничение} & \textbf{PK} & \textbf{FK} \\ \hline
rental\_id & Уникальный ключ сущности аренда & INT &  & + &  \\ \hline
participant\_id & Внешний ключ сущности участник & INT &  &  & + \\ \hline
vehicle\_id & Внешний ключ сущности тр. средства & INT &  &  & + \\ \hline
cost\_amt & Стоимость аренды & REAL & NOT NULL, DEFAULT 0.0, \textgreater{}= 0.0 &  &  \\ \hline
reservation\_dttm & Дата бронирования & TIMESTAMP(0) &  &  &  \\ \hline
return\_dttm & Дата возвращения & TIMESTAMP(0) &  &  &  \\ \hline
distance\_amt & Преодоленное расстояние & REAL & NOT NULL, DEFAULT 0.0, \textgreater{}= 0.0 &  &  \\ \hline
gasoline\_amt & Использованное топливо/электроэнергия & REAL & NOT NULL, DEFAULT 0.0, \textgreater{}= 0 &  &  \\ \hline
\end{tabular}%
}
\caption{Сущность Rental/Аренда}
\label{my-label}
\end{table}


\begin{table}[h!]
\centering
\resizebox{\textwidth}{!}{%
\begin{tabular}{|l|l|c|c|c|c|}
\hline
\multicolumn{6}{|c|}{\textbf{PARTICIPANT\_X\_RENTAL}} \\ \hline
\multicolumn{1}{|c|}{\textbf{Название}} & \multicolumn{1}{c|}{\textbf{Описание}} & \textbf{Тип данных} & \textbf{Ограничение} & \textbf{PK} & \textbf{FK} \\ \hline
participant\_id & Внешний ключ сущности участник & INT &  & + & + \\ \hline
vehicle\_id & Внешний ключ сущности тр. средства & INT &  & + & + \\ \hline
\end{tabular}%
}
\caption{Сущность Participant-Rental}
\label{my-label}
\end{table}

\begin{table}[t]
\centering
\resizebox{\textwidth}{!}{%
\begin{tabular}{|l|l|c|c|c|c|}
\hline
\multicolumn{6}{|c|}{\textbf{RENTAL\_X\_LOCATION}} \\ \hline
\multicolumn{1}{|c|}{\textbf{Название}} & \multicolumn{1}{c|}{\textbf{Описание}} & \textbf{Тип данных} & \textbf{Ограничение} & \textbf{PK} & \textbf{FK} \\ \hline
rental\_id & Внешний ключ сущности аренда & INT &  & + & + \\ \hline
location\_id & Внешний ключ сущности местность & INT &  & + & + \\ \hline
\end{tabular}%
}
\caption{Сущность Rental-Location}
\label{my-label}
\end{table}

\hfill \break
\hfill \break
\hfill \break
\hfill \break
\hfill \break

\end{document}
